\documentclass[a4paper, 12pt]{report}
\usepackage{monapack}
\usepackage{hyperref}

\student{Richard Kropáček}
\trida{B4.I}
\obor{18-20-M/01 Informační technologie}
\bydliste{Mírová 429, 385 01 Vimperk}
\datumNarozeni{11. 11. 2001}
\vedouci{Mgr. Milan Janoušek}
\nazevPrace{Správa povinných prací}
\cisloPrace{1}
\skolniRok{2020/2021}
\reditel{Ing. Jiří Uhlík}

\begin{document}
	
	\titulniStrana
	
	\anotace
	Výstupem této maturitní práce je fungující systém, který má za úkol spravovat povinné práce odevzdané za školní rok žáky. Učitelé následně mohou práce procházet, stahovat, hodnotit.

	\klicovaslova
	C\#, ASP.NET Core, MySQL, WebApp

	\annotation
	The output of this graduation thesis is a functioning system, which has the task of managing the compulsory work submitted for the school year by students. Teachers can then browse, download, evaluate the work.

	\keywords
	C\#, ASP.NET Core, MySQL, WebApp

	\podekovani
	Tímto bych chtěl poděkovat Mgr. Milanu Janouškovi za vedení mé Maturitní práce, cenné rady a odborný dohled. Děkuji také Mgr. Haně Maříkové za pomoc při gramatické kontrole práce.
	
	\obsah
	
	\kapitola{Úvod}
	Pro realizaci této práce jsem se rozhodl využít ASP.NET Core, což jest otevřený framework pro tvorbu webových aplikací vyvynutý společností Microsoft. Pro databázi jsem se rozhodl využít MySQL, což je velice rozšířený otevřený systém řízení báze dat. Webová aplikace je psána pomocí značkovacího jazyku HTML. Pro vizuální část stránky jsem využil Bootstrap 4.
	\kapitola{Teoretický úvod}
		\podkapitola{Programovací jazyk C\#}
	C\# je moderní, vysokoúrovňový objektově orientovaný programocí jazyk vyvinutý firmou Microsoft. Jazyk C\# je typově bezpečný. Tento jazyk umožnuje vývojářům vytvářet mnoho druhů zabezpečených a robusních aplikací, které běži na platformě .NET.\\
	C\# je objektově orientovaný programovací jazyk orientovaný na součásti (component-oriented programming language). Component-oriented programming se zaměřuje na vytváření komponent, které jsou tvořeny často se opakujícími částmi kódu. C\# poskytuje jazykové kontrukce pro přímou podporu těchto konceptů, což z něj dělá přirozený jazyk pro tvorbu a používání těchto softwarových komponent.
		\podkapitola{ASP.NET}
			\podpodkapitola{ASP.NET}
			\podpodkapitola{ASP.NET Core}
		\podkapitola{MySQL}
	\kapitola{Uživatelské rozhraní}
		\podkapitola{Student}
		\podkapitola{Učitel}
		\podkapitola{Administrator}
	\kapitola{Přihlašování a registrace}
	\kapitola{Databáze}

	\kapitola{Závěr}

	\seznamTabulek
	
	\seznamObrazku
	
	\prilohy{
		\kapitola{Příloha}
	}
	
	\literatura{
		
		\kniha{nazevCitace}{Příjmení autora}{Jméno autora}{Název knihy}{Místo vydání}{Nakladatelství}{Rok}{ISBN}
		
		\kvalifikacniprace{nazevCitace}{Příjmení autora}{Jméno autora}{Název práce}{Místo}{Rok}{Druh práce}{Univerzita, Fakulta, Katedra}{Vedoucí diplomové práce jméno}
		
		\url{nazevCitace}{Název stránek}{Titulek}{Stránky}{rok}{datum}{URL odkaz}
		
	}
	
\end{document}