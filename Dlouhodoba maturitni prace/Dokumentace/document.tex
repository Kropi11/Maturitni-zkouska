\documentclass[a4paper, 12pt]{report}
\usepackage{monapack}
\usepackage{hyperref}

\student{Richard Kropáček}
\trida{B4.I}
\obor{18-20-M/01 Informační technologie}
\bydliste{Mírová 429, 385 01 Vimperk}
\datumNarozeni{11. 11. 2001}
\vedouci{Mgr. Milan Janoušek}
\nazevPrace{Správa povinných prací}
\cisloPrace{1}
\skolniRok{2020/2021}
\reditel{Ing. Jiří Uhlík}

\begin{document}
	
	\titulniStrana

	\zadani{31. březen 2021}
	{
		\bod{Proveďte teoretický úvod k problematice řešící návrh webové aplikace pro odevzdávání, třídění a kontrolu odevzdaných povinných prací (výběr vhodné technologie) v rozsahu max. 6 stran.}
		\bod{Realizujete vlastní řešení v následujících bodech (v dokumentaci min. 8 stran).}
			\begin{enumerate}
				\item
				Proveďte rodělení uživatelských skupin a jejich oprávnění.
				\item
				Napište program umožňující správu školních povinných prací.
				\item
				Vyřešte přihlašování uživatelů, pokud možno využijte školní LDAP server.
				\item
				Navrhněte a realizujte propojení s vhodnou databází.
			\end{enumerate}
		\bod{Zpracujte dokumentaci dle metodického návrhu a ppt prezentaci pro účely obhajoby.}
		\bod{Zpracujte materiál pro propagaci výsledků své práce - ppt prezentaci nebo vytvořte poster, nebo informační www stránku, nebo promo video apod.}
	}
	{škola/firma/žák}
	{školy/firmy/žáka}
	{16. 11. 2020}
	
	\anotace

	\annotation
	Výstupem této maturitní práce je fungující systém, který má za úkol spravovat povinné práce odevzdané za školní rok žáky. Učitelé následně mohou práce procházet, stahovat, hodnotit.


	\podekovani
	Tímto bych chtěl poděkovat Mgr. Milanu Janouškovi za vedení mé Maturitní práce, cenné rady a odborný dohled. Děkuji také Mgr. Haně Maříkové za pomoc při gramatické kontrole práce.
	
	\licencniSmlouva{16. 11. 2020}
	
	\obsah
	
	\kapitola{Úvod}
		\podkapitola{C\#}
	C\# je moderní, vysokoúrovňový objektově orientovaný programocí jazyk vyvinutý firmou Microsoft. Jazyk C\# je typově bezpečný. Tento jazyk umožnuje vývojářům vytvářet mnoho druhů zabezpečených a robusních aplikací, které běži na platformě .NET.\\
	C\# je objektově orientovaný programovací jazyk orientovaný na součásti (component-oriented programming language). Component-oriented programming se zaměřuje na vytváření komponent, které jsou tvořeny často se opakujícími částmi kódu. C\# poskytuje jazykové kontrukce pro přímou podporu těchto konceptů, což z něj dělá přirozený jazyk pro tvorbu a používání těchto softwarových komponent.
		\podkapitola{ASP.NET}
			\podpodkapitola{ASP.NET}
			\podpodkapitola{ASP.NET Core}
		\podkapitola{MySQL}
	\kapitola{Uživatelské rozhraní}
		\podkapitola{Student}
		\podkapitola{Učitel}
	\kapitola{Přihlašování a registrace}
	\kapitola{Databáze}

	\kapitola{Závěr}

	\seznamTabulek
	
	\seznamObrazku
	
	\prilohy{
		\kapitola{Příloha}
	}
	
	\literatura{
		
		\kniha{nazevCitace}{Příjmení autora}{Jméno autora}{Název knihy}{Místo vydání}{Nakladatelství}{Rok}{ISBN}
		
		\kvalifikacniprace{nazevCitace}{Příjmení autora}{Jméno autora}{Název práce}{Místo}{Rok}{Druh práce}{Univerzita, Fakulta, Katedra}{Vedoucí diplomové práce jméno}
		
		\url{nazevCitace}{Název stránek}{Titulek}{Stránky}{rok}{datum}{URL odkaz}
		
	}
	
\end{document}