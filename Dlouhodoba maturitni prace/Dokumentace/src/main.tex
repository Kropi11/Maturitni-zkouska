\documentclass[a4paper, 12pt]{report}
\usepackage{monapack}
\usepackage{hyperref}
\usepackage{indentfirst}
\graphicspath{{images/}}
%\usepackage[outputdir=../out]{minted}

\student{Richard Kropáček}
\trida{B4.I}
\obor{18-20-M/01 Informační technologie}
\bydliste{Mírová 429, 385 01 Vimperk}
\datumNarozeni{11. 11. 2001}
\vedouci{Mgr. Milan Janoušek}
\nazevPrace{Správa povinných prací}
\cisloPrace{1}
\skolniRok{2020/2021}
\reditel{Ing. Jiří Uhlík}

\begin{document}

	\titulniStrana
	
	\anotace
	Výstupem mé maturitní práce je fungující systém, jehož úkolem je správa povinných prácí odevzdaných za uplynulý školní rok žáky. Učitelé následně mohou práce procházet, stahovat a hodnotit.

	\klicovaslova
	C\#; ASP.NET Core; Entity Framework Core; HTML; JavaScript; CSS; Bootstrap; MySQL; WebApp

	\annotation
	The output of my graduation thesis is a functioning system, the task of which is the administration of compulsory work submitted by pupils for the past school year. Teachers can then browse, download and evaluate the work.
	\keywords
	C\#; ASP.NET Core; Entity Framework Core; HTML; JavaScript; CSS; Bootstrap; MySQL; WebApp

	\podekovani
	Tímto bych chtěl poděkovat Mgr. Milanu Janouškovi za vedení mé Maturitní práce, cenné rady a odborný dohled. Děkuji také za pomoc všem, kteří se podílěli na gramatické kontrole práce.
	
	\obsah
	
	\kapitola{Úvod}
		Pro realizaci této práce jsem se rozhodl využít ASP.NET Core, což jest otevřený framework pro tvorbu webových aplikací vyvynutý společností Microsoft. Pro 	 databázi jsem se rozhodl využít MySQL, což je velice rozšířený otevřený systém řízení báze dat. Webová aplikace je psána pomocí značkovacího jazyku HTML. Pro vizuální část stránky jsem využil Bootstrap 4.
	\kapitola{Teoretický úvod}
		\podkapitola{Programovací jazyk C\#}
			C\# je moderní, vysokoúrovňový objektově orientovaný programocí jazyk vyvinutý společností Microsoft. Jazyk C\# se řadí mezi typově bezpečné programovací jazyky. To znamená, že nedovoluje provádět operace, které mohou véct k chybám. Tento jazyk umožnuje vývojářům vytvářet mnoho druhů zabezpečených a robusních aplikací, které běži na platformě .NET \footnote{Vývojářská platforma pro vytváření webových, mobilních, desktopových, herních, IoT a dalších aplikací. Je podporovaná v systémech Windows, Linux a macOS.}.\par
			C\# je zároveň objektově orientovaný programovací jazyk orientovaný na součásti \footnote{Component-oriented programming language}. To znamená, že se zaměřuje na vytváření komponent, které jsou tvořeny často se opakujícími částmi kódu. C\# také poskytuje jazykové kontrukce pro přímou podporu těchto konceptů, což z něj dělá přirozený jazyk pro tvorbu a používání těchto softwarových komponent.
	\cite{CSharp}

		\podkapitola{ASP.NET, ASP.NET CORE}
			\obrazek{ASP.NET Core a ASP.NET}{Porovnání ASP.NET Core a ASP.NET \cite{ASPNETCORE_ASPNET}}{12cm}{aspnetcore_aspnet}
			ASP.NET je webový framework obsahující sadu knihoven, které obsahují hotová řešení mnoha základních problémů, které ve webových technologiích vyvstávají. Může se jednat např. o bezpečnost, autentifikaci uživatele, práci s databázemi apod.\par
			Tato technologie je založena na architektuře klient-server. Výstupem ASP.NET aplikace je HTML stránka. ASP.NET tedy běží na serveru a reaguje na dotazy uživatele/klienta. Pro tvorbu ASP.NET aplikace je potřeba znalost především programovacího jazyku C\# a značkovacího jazyku HTML.
			\cite{ASP.NET_Lekce1}
			\podpodkapitola{ASP.NET}
				ASP.NET je webový framework s otevřeným zdrojovým kódem, vytvořený společností Microsoft, pro vytváření moderních webových aplikací. ASP.NET rozšiřuje platformu .NET o nástroje a knihovny určené právě pro vytváření webových aplikací.
				\cite{ASP.NET}
			\podpodkapitola{ASP.NET Core}
				ASP.NET Core je open-source a multiplatformní verze ASP.NET. Tato platforma je navržena tak, aby umožnila rychlé vyvíjení runtime komponent, API, překladačů apod. a zároveň poskytovala stabilní a podporovanou platformu pro udržení běhu aplikací.\par
				Aplikace v ASP.NET Core lze, oproti dřívější verzi APS.NET Windows-only version, vyvíjet a spouštět v systémech Windows, Linux, macOS a Docker.
				\cite{ASP.NET_Core}

		\podkapitola{EF6 a EF Core}
			Entity Framework je Object–relational mapping (ORM) \footnote{Objektově-relační mapování}. To znamená, že se databázové tabulky přímo mapují na C\# třídy. V projektu následně pracujeme pouze s objekty a framework za nás na pozadí generuje SQL dotazy. Díky tomu je výsledná aplikace tvořena především pomocí objektů.
			\cite{ASP.NET_Lekce8}
			\podpodkapitola{Entity Framework 6}
				Entity Framework 6 (EF6) je ORM, primárně navržený pro .NET Framework, ale zároveň s podporou pro .NET Core. EF6 je stabilní, podporovaný produkt, ale již se aktivně nevyvíjí.
				\cite{EF6_EFCore}
			\podpodkapitola{Entity Framework Core}
				Entity Framework Core (EF Core) je moderní ORM pro .NET. Podporuje dotazy LINQ, sledování změn, aktualizace a migrace schématu.\par
				EF Core pracuje s SQL Server/SQL Azure, SQLite, Azure Cosmos DB, MySQL, PostgreSQL a mnoha dalšími databázemi prostřednictvím modelu modulů plug-in zprostředkovatele databáze.
				\cite{EF6_EFCore}

		\podkapitola{MVC}
		Model-View-Controller (MVC) je návrhový vzor používaný k rozdělení webové aplikace na 3 komponenty.
		\begin{itemize}
			\item Model - Data
			\item View - Uživatelské rozhraní
			\item Controller- Logická část aplikace
		\end{itemize}
		Pomocí vzoru MVC pro webové aplikace jsou požadavky směrovány na controller, který je zodpovědný za práci s modelem. Model provádí akce a načítá data za databáze. Controller poté zvolí zobrazení (view), které se má zobrazit, a poskytne mu model. View už pouze vykreslí stránku na základě dat získaných z modelu.
		\cite{MVC}
				\podpodkapitola{Model}
				Jedná se o dynamickou datovou strukturu aplikace, nezávislou na uživatelském rozhraní. Jejím úkolem je spravovat data, pravida a logiku aplikace. \par
				V ASP.NET Core MVC má model podobu třídy, která je provázaná s databází. Pomocí atributů definuje pravidla, která se aplikují na klienta a server.
				\cite{MVC_Wiki_EN}
				\podpodkapitola{View}
				Převádí data reprezentovaná modelem do podoby vhodné k interaktivní prezentaci uživateli.\cite{MVC_Wiki_CZ} Může se jednat o jakoukoliv reprezentaci dat (graf, diagram, tabulka, atd.). \par
				V ASP.NET Core se využívá syntaxe Razor. Ta poskytuje jednoduchý, čistý a lehký způsob vykreslení obsahu HTML stránek na základě view. Razor umožňuje vykreslit stránku pomocí C\# a vytvářet webové stránky plně kompatibilní s HTML5.
				\cite{MVC_Wiki_EN}
				\podpodkapitola{Controller}
				Controller reaguje na vstup uživatele a provádí interakce s objekty datového modelu. Zjednodušeně to znamená, že řadič přijme vstup, ověří jej a následně ho předá modelu.
				\cite{MVC_Wiki_EN}
	\podkapitola{Razor}

	\podkapitola{MySQL}
		MySQL je otevřený systém řízení báze dat uplatňující relační databázový model. Jedná se o multiplatformní databázi. Komunikace s databází probíhá pomocí dotazovacího jazyka SQL.
		\cite{MySQL_Wiki_CZ}
	\kapitola{Uživatelské rozhraní}
		\podkapitola{Student}
		\podkapitola{Učitel}
		\podkapitola{Administrator}
	\kapitola{Přihlašování a registrace}
	\kapitola{Databáze}

	\kapitola{Závěr}

	\seznamTabulek
	
	\seznamObrazku
	

	\bibliographystyle{czechiso}
	\bibliography{main}

	\prilohy{
		\kapitola{Příloha}
	}

\end{document}